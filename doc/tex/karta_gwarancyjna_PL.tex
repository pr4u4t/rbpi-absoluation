\documentclass[10pt,a4paper]{article}

\usepackage{polski}
\usepackage[utf8]{inputenc}	
\usepackage[T1]{fontenc}
\usepackage{lmodern}
\usepackage{graphicx}
\usepackage{listings}
\usepackage{enumitem}
\usepackage{titlesec}
\usepackage{hyperref}
\usepackage{array}
\usepackage[margin=1in]{geometry}

%\usepackage{booktabs}

%variables
\newcommand{\shopname}{SHOP NAME}
\newcommand{\companyname}{COMPANY NAME}
\newcommand{\regon}{REGON}
\newcommand{\nip}{NIP}
\newcommand{\httpaddr}{SITE ADDRESS}
\newcommand{\address}{ADDRESS}
\newcommand{\mail}{MAIL ADDRESS}
\newcommand{\phone}{PHONE NUMBER}
\newcommand{\currency}{CURRENCY}

% use paragraph sign with numbering and break section title below
\titleformat{\section}[display]
	{\normalfont\Large\bfseries\filcenter}{\S\thesection}{1em}{}

% display subsection numbering without newline
\titleformat{\subsection}[runin]
	{\fontsize{12}{15}\filcenter\sffamily}
	{\thesubsection}{1em}{}

% command to set text left and right margin
\def\changehmargin#1#2{\list{}{\rightmargin#2\leftmargin#1}\item[]}
\let\endchangehmargin=\endlist 

% pdf form input adjustment
\newlength{\LabelWidth}%
\setlength{\LabelWidth}{1.3in}%
\newcommand*{\AdjustInputSize}[2][l]{\makebox[\LabelWidth][#1]{#2}}%
\def\textfieldwidth{2.25in}%

%define new column type
\newcolumntype{L}[1]{>{\raggedright\let\newline\\\arraybackslash\hspace{0pt}}m{#1}}
\newcolumntype{C}[1]{>{\centering\let\newline\\\arraybackslash\hspace{0pt}}m{#1}}
\newcolumntype{R}[1]{>{\raggedleft\let\newline\\\arraybackslash\hspace{0pt}}m{#1}}

%title autor etc

\title{OGÓLNE WARUNKI GWARANCJI} % The article title
\author{}
\date{} 

%----------------------------------------------------------------------------------------

\begin{document}
	\maketitle

	\section{ZAKRES ZASTOSOWANIA}

	\subsection{Ogólne Warunki Gwarancji (dalej OWG) stanowią integralną część Umów Sprzedaży oraz związanych z nimi umów o świadczenie
usług zawieranych pomiędzy COMODO Group Spółka z ograniczoną odpowiedzialnością Spółka komandytowa a Kupującymi
oferowanych przez nią produktów, o ile umowy te nie stanowią inaczej. Użyte w dalszej części niniejszych OWG określenia oznaczają:}

\begin{itemize}

\item \shopname \companyname \nip \regon

\item „Kupujący” - kontrahenta dokonującego od \shopname zakupów produktów lub usług. Niniejsze OWG stosuje się tylko do
kontrahentów (przedsiębiorców art. 43 1k.c.) nie będących konsumentami w rozumieniu art. 22 1Kodeksu Cywilnego.

\item „Strony” - \shopname i Kupującego

\item „OWG” - niniejsze Ogólne Warunki Gwarancji \companyname

\item „Gwarancja Producenta” – warunki gwarancyjne oferowane niezależenie od niniejszych OWG bezpośrednio przez Producenta.

\item „Produkt” - produkty, towary i usługi stanowiące przedmiot statutowej działalności gospodarczej \shopname i w powyższym zakresie
objęte gwarancją na terenie Polski, w szczególności: meble, krzesła, 
wykładziny, itp.

\item „Produkty małogabarytowe” – Produkty o gabarytach nie przekraczających wymiarów: szerokość 80cm, głębokość 80cm, wysokość
140cm, oraz wagi 50kg, których transport nadmiernie nie obciąża Kupującego, np. krzesło, kontener z szufladami, mała szafka,
akcesoria biurowe, itp.

\item „Produkty wielkogabarytowe” – Produkty o gabarytach przekraczających wymiary i wagę Produktów małogabarytowych lub
Produkty niezależnie od ich wymiarów i wagi, które zostały przez \shopname zmontowane w zestaw, np.: biurko z pomocnikiem,
zestaw połączonych ze sobą szaf, zabudowa wnęki, meble kuchenne, lada recepcyjna, itp.

\item „Producent” – wytwórca produktów, towarów, sprzedawanych za pośrednictwem \shopname

\item „Umowa Sprzedaży” – dokument na podstawie, którego \shopname realizował dostawę Produktów dla Kupującego.

\item „Reklamacja” – pisemnie zgłoszona wada lub uszkodzenie Produktu.

\item „Miejsce montażu” – siedziba Kupującego lub inne miejsce, które wyszczególnione było w Umowie Sprzedaży Produktu. 

\end{itemize}

\subsection{Zgodnie z niniejszym OWG \shopname udziela Kupującemu gwarancji na wszystkie sprzedawane przez siebie Produkty, zapewnia
sprawne działanie oferowanych Produktów pod warunkiem korzystania z nich zgodnie z przeznaczeniem i warunkami eksploatacji
określonymi w dokumentacji Producenta oraz niniejszej OWG.}

\subsection{Zgodnie z art. 558 § 1 Kodeksu cywilnego rękojmia za Produkt jest wyłączona.}

\section{WARUNKI EKSPLOATACJI} 

\subsection{Produkty powinny być użytkowane w pomieszczeniach suchych i przewiewnych, o stałej temperaturze i wilgotności.}

\subsection{Produkty należy chronić przed bezpośrednim działaniem promieni słonecznych
Produkty skórzane nie powinny stać blisko grzejnika lub innych źródeł ciepła. Minimalna odległość od źródła ciepła powinna wynosić 30 cm.}

\subsection{Konserwacji skóry należy dokonywać specjalnymi środkami pielęgnacyjnymi, które tworzą powłokę ochronną, odporną na wodę i substancje oleiste.}

\subsection{Konserwacji mebli wykonanych w okleinach sztucznych należy dokonywać poprzez czyszczenie miękką ściereczką lub gąbką przy użyciu preparatów myjących i konserwujących powierzchnie drewnopochodne, lakierowane i laminowane (np. płyn do mycia szyb bez amoniaku).}

\subsection{Meble w okleinach naturalnych należy czyścić bardzo delikatnie, miękką ściereczką nasączoną (nigdy w formie rozlewania jej na
powierzchni mebla) 2-procentowym roztworem alkoholu (np. płyn do mycia szyb bez amoniaku, roztwór mydła w proporcji 250 ml
wiórków mydlanych na 1 litr ciepłej wody oraz mleczka do konserwacji forniru, z wyłączeniem tłustych środków zapobiegających
osiadaniu kurzu np. Pronto przeciw kurzowi), zgodnie z rysunkiem słojów.
Światło ma wpływ na fornir okleiny naturalnej. Przez pierwsze 6-8 tygodni, gdy mebel początkowo pochłania światło z otoczenia, nie
należy stawiać na powierzchniach żadnych przedmiotów ani ozdób. Nierównomierne pochłanianie światła może doprowadzić do
powstawania trwałych odbarwień, np. w kształcie okręgów.}

\subsection{Podczas eksploatacji mebli w okleinach naturalnych zaleca się stosowanie podkładek na blatach, w szczególności pod laptopa,
klawiaturę, mysz komputerową oraz w miejscach szczególnie narażonych na uszkodzenia mechaniczne.}

\subsection{Przemieszczanie mebli może być dokonywane jedynie poprzez ich przenoszenie (UWAGA: mebli nie wolno przesuwać po podłodze).}

\subsection{Zestawy szaf i biurek połączone ze sobą konstrukcyjnie przed przemieszczeniem należy zdemontować. Wskazane jest, aby w takim
przypadku demontażu dokonał serwis \shopname.}

\subsection{Kupujący dokonuje czyszczenia i konserwacji produktów we własnym zakresie i na własny koszt.}

\section{ZAKRES GWARANCJI}

\subsection{\shopname udziela Kupującemu gwarancji na wszystkie sprzedawane przez siebie Produkty, zapewnia sprawne działanie oferowanych
produktów pod warunkiem korzystania z nich zgodnie z przeznaczeniem i warunkami eksploatacji określonymi w dokumentacji.}

\subsection{W okresie trwania gwarancji \shopname zobowiązany jest bezpłatnie dostarczyć części zamienne lub naprawić wadliwe Produkty.}

\subsection{\shopname odpowiada przed Kupującym wyłącznie za wady fizyczne powstałe z przyczyn tkwiących w sprzedanym Produkcie.Gwarancja nie są objęte wady powstałe z innych przyczyn, a szczególnie w wyniku:}

\begin{itemize}

\item uszkodzeń mechanicznych i wywołanych nimi wad
uszkodzeń powstałych podczas transportu i przeładunku (nie dotyczy przewozu wykonanego przez \shopname)
\item uszkodzeń wynikających z eksploatacji Produktów w warunkach, które nie odpowiadają normalnym warunkom eksploatacyjnym oraz
wynikających z nieprawidłowej konserwacji
\item samowolnych przeróbek lub zmian konstrukcyjnych
nieprawidłowego montażu, konserwacji, magazynowania Produktu przez Kupującego
\item uszkodzeń wynikłych ze zdarzeń losowych, czynników noszących znamiona siły wyższej (pożar, powódź, wyładowania atmosferyczne
itp.)
\item wykonywania samodzielnych napraw


\end{itemize}

\subsection{Gwarancja nie obejmuje części podlegających normalnemu zużyciu Produktu wynikających z jego poprawnej eksploatacji}

\subsection{Gwarancją nie są objęte zmiany spowodowane działaniem promieni słonecznych oraz naturalne cechy surowca, takie jak: różnice w
odcieniach i usłojeniu mebli fornirowanych, różnice w fakturze i odcieniach oraz blizny i ślady ukłuć owadów w Produktach pokrytych skórą.}

\subsection{Gwarancja nie obejmuje Produktu, którego na podstawie przedłożonych dokumentów i cech znamionowych nie można zidentyfikować jako Produktu zakupionego u \shopname.}

\section{OKRES GWARANCJI}

\subsection{Okres gwarancji na Produkty oferowane przez \shopname liczony jest od daty sprzedaży i wynosi standardowo 24 miesiące.}
\subsection{Strony w Umowie Sprzedaży mogą ustalić inne okresy gwarancji.}

\subsection{\shopname udziela Klientowi gwarancji na okres podany powyżej na podstawie faktury VAT lub Umowy Sprzedaży Produktu. Na życzenie \shopname wyda Klientowi Kartę gwarancyjną.}

\subsection{Jeżeli Producent oferuje dłuższy okres gwarancyjny Kupujący ma prawo, wedle własnego uznania skorzystać z niniejszych OWG lub
Gwarancji Producenta z zastrzeżeniem ust. 5.13 i 5.14}

\section{ZGŁOSZENIE I PROCEDURA GWARANCYJNA}

\subsection{Podstawą przyjęcia reklamacji do rozpatrzenia jest spełnienie łącznie następujących warunków:}

\begin{itemize}

\item zgłoszenia Reklamacji przez Kupującego w formie pisemnej i przesłania listem poleconym lub na adres e-mail
\mail, ewentualnie za pośrednictwem faxu (tel. \phone), na odpowiednim Formularzu reklamacyjnym,
zawierającym: nazwę towaru, numer katalogowy, datę zakupu, szczegółowy opis uszkodzenia wraz z dodatkowymi informacjami
dotyczącymi powstania wad Produktu oraz załączonymi zdjęciami wadliwego Produktu. Wzór formularza dostępny jest na stronie
internetowej \httpaddr lub w siedzibie \shopname.

\item okazania faktury zakupu lub Umowy Sprzedaży reklamowanego produktu.

\item dostarczenia osobistego lub za pośrednictwem firmy kurierskiej reklamowanego Produktu do siedziby \shopname (dotyczy Produktów
małogabarytowych) lub umożliwienie, na każdą prośbę \shopname, dostępu do Produktów wielkogabarytowych, w miejscu ich
montażu.

\end{itemize}

\subsection{Wady lub uszkodzenia Produktu ujawnione w okresie gwarancji powinny zostać zgłoszone \shopname niezwłocznie, nie później
jednak niż 7 dni od daty ich ujawnienia.}

\subsection{Produkt, w którym stwierdzono wadę powinien zostać niezwłocznie wyłączony z użytkowania pod rygorem utraty gwarancji.}

\subsection{\shopname w ciągu 14 dni roboczych, od momentu zgłoszenia Reklamacji, przeprowadza badania wyrobu i w przypadku stwierdzenia, że
wada nastąpiła z przyczyn tkwiących w sprzedanym Produkcie - potwierdza przyjęcie zgłoszenia reklamacyjnego oraz określa sposób i
zakres naprawy lub wymiany wadliwego towaru.}

\subsection{Jeżeli \shopname nie jest w stanie stwierdzić, na podstawie złożonej Reklamacji, czy wada nastąpiła z przyczyn tkwiących w sprzedanym
Produkcie, odsyła reklamowany Produkt do siedziby Producenta. W przypadku odrzucenia Reklamacji przez Producenta, Produkt
odsyłany jest do miejsca montażu na koszt i ryzyko Kupującego. \shopname zastrzega sobie również prawo do obciążenia Kupującego
kosztami manipulacyjnymi związanymi z przeprowadzeniem ekspertyzy Produktu (gdyby była konieczna), jeśli reklamowany Produkt
okaże się sprawny lub uszkodzenie nie było objęte gwarancją.}

\subsection{Wady produktu ujawnione w okresie gwarancji będą usuwane bezpłatnie w terminie nie dłuższym niż 14 dni roboczych od daty
przyjęcia pisemnego zgłoszenia. W przypadku, gdy zachodzi konieczność dokonania skomplikowanej naprawy, a także gdy konieczne
jest sprowadzenie części zamiennych czy podzespołów z zagranicy, \shopname zastrzega sobie możliwość przedłużenia terminu naprawy
nie dłużej jednak niż o 21 dni roboczych, po uprzednim poinformowaniu Kupującego.}

\subsection{Okres naprawy gwarancyjnej reklamowanego wyrobu przedłuża odpowiednio czas udzielonej gwarancji na dany Produkt.}

\subsection{W przypadku gdy usunięcie wady nie jest możliwe lub wiązałoby się to z nadmiernymi kosztami, \shopname może wymienić wyrób na
wolny od wad lub zwrócić uiszczoną zapłatę. W wypadku wymiany rzeczy na wolną od wad w ramach realizacji świadczeń
gwarancyjnych o których mowa powyżej, termin gwarancji dla wymienionej rzeczy biegnie na nowo od chwili dostarczenia rzeczy
wolnej od wad.}

\subsection{Produkty małogabarytowe należy, po uprzednim ustaleniu z \shopname, odesłać na jego adres, przy czym koszty i ryzyko przesyłki
ponosi Kupujący. Uznanie roszczeń gwarancyjnych Kupującego będzie równoznaczne z naprawą Produktu lub wymianą Produktu na
wolny od wad i zwrotem kosztów przesyłki poniesionych przez Kupującego}

\subsection{Produkty wielkogabarytowe, w zależności od rodzaju wady, mogą zostać naprawione przez \shopname na miejscu montażu lub, w
przypadku braku takiej możliwości, wysłane na ryzyko i koszt \shopname do siedziby Producenta.}

\subsection{Zarówno w przypadku naprawy dokonywanej przez \shopname w miejscu montażu, jak i konieczności dokonania demontażu przygotowania reklamowanego towaru do wysyłki do siedziby Producenta, Kupujący zobowiązany jest do zapewnienia swobodnego
dostępu do Produktu, usunięcia z jego powierzchni lub wnętrza wszelkich rzeczy Kupującego np. sprzętu biurowego, dokumentów,
akcesoriów biurowych itp. W innym przypadku serwisant ma prawo domówić działań serwisowych.}

\subsection{Złożenie Reklamacji przez Kupującego nie zwalnia go z obowiązku zapłaty za reklamowy Produkt. \shopname zastrzega sobie prawo
wstrzymania procedury gwarancyjnej w przypadku gdy Kupujący zalega z płatnościami za faktury przeterminowane dłużej niż 7 dni.}

\subsection{Niniejsze OWG nie wyłączają uprawnień Kupującego do skorzystania z Gwarancji Producenta oferowanych bezpośrednio przez
fabrykę wytwarzającą zakupiony produkt. W przypadku zgłoszenia reklamacji bezpośrednio do Producenta, Kupującego obowiązują
wszelkie wymogi formalne, eksploatacyjne, okresy gwarancji, procedury reklamacyjne opisane na Kartach Gwarancyjnych
Producenta, załączonych do poszczególnych Produktów lub dostępnych na stronach internetowych Producenta.}

\subsection{W przypadku skorzystania przez Kupującego z Gwarancji Producenta gwarancja udzielana przez \shopname 
przestaje obowiązywać.}

\subsection{W sprawach nieuregulowanych niniejszym OWG mają zastosowanie postanowienia Kodeksu Cywilnego.}

\end{document}
