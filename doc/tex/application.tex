\documentclass[
12pt, % Main document font size
a4paper
%, % Paper type, use 'letterpaper' for US Letter paper
%oneside, % One page layout (no page indentation)
%twoside, % Two page layout (page indentation for binding and different headers)
%headinclude,footinclude, % Extra spacing for the header and footer
%BCOR5mm, % Binding correction
]{report}

\usepackage[font=small,labelfont=bf,justification=raggedright,format=hang,singlelinecheck=off,textfont=it]{caption}
%\usepackage{subcaption}
\usepackage{indentfirst}
\usepackage{longtable}
\usepackage{tabu}
\usepackage{titlepic}
\usepackage{geometry}
\geometry{margin=1cm}
\usepackage{mathtools}
\usepackage{polski}
\usepackage[utf8]{inputenc}
\usepackage{booktabs}
\usepackage[T1]{fontenc}
\usepackage{lmodern}
\usepackage{fancyhdr}
\pagestyle{fancy}
\pagestyle{headings}
\usepackage{float}
\usepackage{gensymb}
\usepackage{graphicx}
\usepackage{listings}
\usepackage{enumitem}
\usepackage{titlesec}
\usepackage{pdfpages}
\usepackage{wallpaper}
\usepackage{afterpage}
\usepackage{xfrac}
\usepackage{hyperref}
\hypersetup{
    colorlinks,
    citecolor=black,
    filecolor=black,
    linkcolor=black,
    urlcolor=black
}
\usepackage{cleveref}

\newcommand\blankpage{%
    \null
	\ClearWallPaper
    \thispagestyle{empty}%
    \addtocounter{page}{-1}%
    \newpage
	\ULCornerWallPaper{1}{head}}

%\captionsetup{}

%\usepackage{showframe}


\newcounter{magicrownumbers}
\newcommand\rownumber{\stepcounter{magicrownumbers}\arabic{magicrownumbers}}

%variables
\newcommand{\shopname}{SHOP NAME}
\newcommand{\companyname}{COMPANY NAME}
\newcommand{\regon}{REGON}
\newcommand{\nip}{NIP}
\newcommand{\httpaddr}{SITE ADDRESS}
\newcommand{\address}{ADDRESS}
\newcommand{\mail}{MAIL ADDRESS}
\newcommand{\phone}{PHONE NUMBER}
\newcommand{\currency}{CURRENCY}

\setlistdepth{9}
\graphicspath{ {images/} }

\title{Zgłoszenie tematu pracy Adrianna Mickiewicz}
\author{Adrianna Mickiewicz <209342@student.pwr.edu.pl>}
%\titlepic{\includegraphics[width=128px]{logo.png}}

%definitions
\newif\ifpersonal
\personaltrue % comment out to hide answers

%styling
\titlespacing*{\subparagraph}{1em}{0pt}{0pt}
\titleformat{\subparagraph}[runin]
{\normalfont\normalsize}{\thesubparagraph}{1em}{}

%\ULCornerWallPaper{1}{head}
%\LLCornerWallPaper{1}{foot}

\begin{document}	
	\thispagestyle{empty}
	\begin{center}
		\noindent\textbf{\Large{Projekt systemu monitorującego ruch zasobów z wykorzystaniem mikrokontrolera i identyfikacji radiowej RFiD}}
		\noindent Project of a system for monitoring the movement of resources using microcontroller and radio frequency identification RFiD.
	\end{center}
	\vspace{1em}
	
	\noindent\textbf{Cel i zadania pracy:}
	%\begin{itemize}
	%	\item 	
			Celem pracy jest zaprojektowanie systemu monitorującego zasoby oznaczone znacznikami RFiD. 
			System umożliwa identyfikację, zabezpieczenie oraz kontrolę nad ruchem zasobów.
			Platforma znajduje zastosowanie w systemach produkcyjnych, logistycznych oraz kontroli zasobów przedsiebiorstw. 
			Praca zawiera projekt urządzenia kontrolnego, monitora systemu i centrali, oraz integrację komponentów systemu poprzez sieć komputerową.
	%\end{itemize}
        \vspace{1em}
 
         \noindent\textbf{Zadania:}
         \begin{itemize}
                \item Zaprojektowanie urządzenia kontrolnego z gotowych podzespołów elektronicznych (mikrokontroler, czytnik rfid),
		\item oprogramowanie mikrokontrolera realizujące identyfikację znaczników,
		\item zaprojektowanie i oprogramowanie urządzenia monitorującego zdarzenia w systemie,
		\item wykonanie centrali systemu wykorzystującej system bazodanowy, 
		\item projekt komunikacji między składowymi systemu,
		\item budowa urządzenia kontrolnego,
		\item budowa urządzenia monitorującego,
		\item testy systemu;
         \end{itemize}

\end{document}

