	\subsection{Wstęp}
		\par Przed założeniem własnej firmy, każdy przyszły przedsiębiorca musi zdobyć środki finansowe na jej funkcjonowanie. Sposób pozyskiwania tych środków zależy od wielkości firmy, wzrostu, etapu jej rozwoju. Źródła finansowania przedsiębiorstwa są różne. Może być to nasz własny kapitał albo kapitał obcy. Jeden i drugi ma swoje plusy i minusy. Wkład własny jest z jednej strony najbezpieczniejszy, ale najczęściej jest bardzo niski. Kapitał obcy czyli leasing, kredyty bankowe, dotacje i subwencje itp., zawsze trzeba zwrócić.
		
	\subsection{Źródła finansowania przedsiębiorstwa}
		\par Do prowadzenia sklepu internetowego potrzebuję właściwego oprogramowania oraz platformy sprzętowej, które zapewnią mi prowadzenie działalności na najwyższym poziomie. Dużym nacisk kładę na bezpieczeństwo zarówno sklepu jak i klientów.
		
		\par Do sfinansowania tego przedsięwzięcia potrzebne są mi środki finansowe, które umożliwią mi zakup drogiego sprzętu komputerowego. Głównym źródłem finansowania będzie tu dotacja ze środków unijnych w ramach projektu "Akademia przedsiębiorczości kobiet"". Pełna kwota dotacji to 24000 zł dofinansowania jednorazowego. Dodatkowo mogę otrzymać comiesięczne wsparcie pomostowe w wysokości 1350 zł na pokrycie codziennych kosztów związanych z prowadzeniem działalności.
		
		\par Posiadam również kapitał własny w wysokości 8 tysięcy złotych, które mogę przeznaczyć na reklamę, zakup mebli biurowych itp. ,oraz własnego laptopa niezbędnego do prowadzenia sklepu internetowego o wartości 2200 zł. 
		
		\par Warto również zaznaczyć, że posiadam drogie narzędzia programowe, gwarantujące przede wszystkim niezależność, bezpieczeństwo sklepu jak i klientów, oraz nienaganne działanie platformy. Są to:
		
		\begin{itemize}
		
			\item Oprogramowanie sklepu internetowego bazujące na platformie Drupal
				
			\item Oprogramowanie serwera stron internetowych
				
			\item Oprogramowanie serwera wiadomości e-mail
				
			\item Centrala telefoniczna PBX
				
			\item System zarządzania bazujący na platformie Redmine
				
				
		\end{itemize}	
				
				
				\par W prognozie na 3 lata uwzględniłam również zatrudnienie pracowników. Mogę to zrobić poprzez urząd pracy, a tym samym starać się o wsparcie finansowe, związane z tym zatrudnieniem. Mogę starać się o dofinansowanie wynagrodzenia nowo zatrudnionego, jego składek ubezpieczenia społecznego, jak również refundację doposażenia jego stanowiska pracy. Takie działanie przynosi obustronne korzyści- osoby bezrobotne mają szanse na dobrą pracę, a ja jako pracodawca zminimalizuje koszty związane z zatrudnieniem nowego pracownika.	
