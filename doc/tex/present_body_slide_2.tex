\begin{frame}
    \frametitle{Introduction}
    \begin{minipage}{\textwidth}        
        Although Web pages may appear fancy and informative to human readers, they can be
        highly unstructured and lack a predefined schema, type, or pattern. Thus it is difficult for
        computers to understand the semantic meaning of diverse Web pages and structure them
        in an organized way for systematic information retrieval and data mining. Web services
        that provide keyword-based searches without understanding the context behind the Web
        pages can only offer limited help to users. For example, a Web search based on a single
        keyword may return hundreds of Web page pointers containing the keyword, but most
        of the pointers will be very weakly related to what the user wants to find. Data mining
        can often provide additional help here than Web search services. For example, authoritative
        Web page analysis based on linkages among Web pages can help rank Web pages based on their importance, influence, and topics. Automated Web page clustering and
        classification help group and arrange Web pages in a multidimensional manner based
        on their contents. Web community analysis helps identify hidden Web social networks
        and communities and observe their evolution. Web mining is the development of scalable
        and effective Web data analysis and mining methods. It may help us learn about the
        distribution of information on the Web in general, characterize and classify Web pages,
        and uncover Web dynamics and the association and other relationships among different
        Web pages, users, communities, and Web-based activities.
    \end{minipage}
\end{frame}
