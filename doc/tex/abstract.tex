\documentclass[
12pt, % Main document font size
a4paper
%, % Paper type, use 'letterpaper' for US Letter paper
%oneside, % One page layout (no page indentation)
%twoside, % Two page layout (page indentation for binding and different headers)
%headinclude,footinclude, % Extra spacing for the header and footer
%BCOR5mm, % Binding correction
]{report}

\usepackage[font=small,labelfont=bf,justification=raggedright,format=hang,singlelinecheck=off,textfont=it]{caption}
%\usepackage{subcaption}
\usepackage{indentfirst}
\usepackage{longtable}
\usepackage{tabu}
\usepackage{titlepic}
\usepackage{geometry}
\geometry{margin=1in,top=1.5in,bottom=1.5in}
\usepackage{mathtools}
\usepackage{polski}
\usepackage[utf8]{inputenc}
\usepackage{booktabs}
\usepackage[T1]{fontenc}
\usepackage{lmodern}
\usepackage{fancyhdr}
\pagestyle{fancy}
\pagestyle{headings}
\usepackage{float}
\usepackage{graphicx}
\usepackage{listings}
\usepackage{enumitem}
\usepackage{titlesec}
\usepackage{pdfpages}
\usepackage{wallpaper}
\usepackage{cleveref}
\usepackage{afterpage}

\usepackage{hyperref}
\hypersetup{
    colorlinks,
    citecolor=black,
    filecolor=black,
    linkcolor=black,
    urlcolor=black
}

\newcommand\blankpage{%
    \null
	\ClearWallPaper
    \thispagestyle{empty}%
    \addtocounter{page}{-1}%
    \newpage
	\ULCornerWallPaper{1}{head}}

%\captionsetup{}

%\usepackage{showframe}


\newcounter{magicrownumbers}
\newcommand\rownumber{\stepcounter{magicrownumbers}\arabic{magicrownumbers}}

%variables
\newcommand{\shopname}{SHOP NAME}
\newcommand{\companyname}{COMPANY NAME}
\newcommand{\regon}{REGON}
\newcommand{\nip}{NIP}
\newcommand{\httpaddr}{SITE ADDRESS}
\newcommand{\address}{ADDRESS}
\newcommand{\mail}{MAIL ADDRESS}
\newcommand{\phone}{PHONE NUMBER}
\newcommand{\currency}{CURRENCY}

\setlistdepth{9}
\graphicspath{ {images/} }

\title{Biznes plan sklep-meblowy.org}
\author{Kamila Klonowska <kamila.klon@sklep-meblowy.org>}
\titlepic{\includegraphics[width=128px]{logo.png}}

%definitions
\newif\ifpersonal
\personaltrue % comment out to hide answers

%styling
\titlespacing*{\subparagraph}{1em}{0pt}{0pt}
\titleformat{\subparagraph}[runin]
{\normalfont\normalsize}{\thesubparagraph}{1em}{}

\ULCornerWallPaper{1}{head}
%\LLCornerWallPaper{1}{foot}

\begin{document}			


	%strona tytulowa
	\begin{titlepage}
	\clearpage\maketitle
	\thispagestyle{empty}
\end{titlepage}

	\afterpage{\blankpage}
	
	%spis tresci
	% Set the depth of the table of contents to show sections and subsections only
%\setcounter{tocdepth}{2} 
\newpage 
\tableofcontents % Print the table of contents
\newpage

	
Skuteczne i bezawaryjne działanie sklepu internetowego zależy od odpowiedniego sprzętu. Podstawowymi środkami umożliwiającymi uruchomienie sklepu internetowego są oprogramowanie oraz sprzęt komputerowy i sieciowy. Urządzenia te umożliwią budowę spójnego systemu wymiany informacji i dokumentów. W skład systemu wchodzą:
2 komputery klasy serwerowej
2 komputery klasy dysku sieciowego
Switch umożliwiający połączenie urządzeń w sieć
Acces Point punkt dostępu dla klientów radiowych - WIFI 
Urządzenie wielofunkcyjne (skaner,ksero,drukarka,fax)
Telefon VOIP 
Kasa fiskalna
System zarządzania
Sklep internetowy
Zastosowanie 2 serwerów i 2 dysków sieciowych umożliwia osiągnięcie niezawodności platformy na poziomie 96\% w okresie 3 lat. Sklep internetowy umożliwia klientom zapoznanie się z oferowanymi produktami. System zarządzania odpowiada za przechowywanie dokumentów, zleceń produkcji, dokumentów celnych oraz dowodów wysyłki. Rozdzielenie tych systemów zapewnia bezpieczeństwo danych pochodzących z obsługi sklepu. 
Komputer składa się z następujących elementów:
Procesor (CPU)
Płyta główna (MOBO)
Pamięć RAM (RAM)
Dysk twardy (HDD)
Zasilacz (PSU)
Karta graficzna (GPU)
Karta sieciowa (ETH)
Obudowa 
Na serwerach zostaną uruchomione odpowiednio sklep internetowy, oraz system zarządzania są to wyspecjalizowane zespoły programów umożliwiające klientom zapoznanie się z ofertą sklepu, filtrowanie produktów w zależności od potrzeb, zakup itp. System zarządzania będzie odpowiedzialny za prawidłowe przeprowadzenie procesów sprzedaży i ewentualnych reklamacji. 
Będą w nim przechowywane dokumenty związane z produkcją, płatnością oraz wysyłką towaru. Takie rozwiązanie umożliwi sprawne zarządzanie obiegiem dokumentów w firmie. Ze względu na dużą złożoność systemu zarządzania i sklepu internetowego muszą być one usytuowane na dwóch osobnych urządzeniach. Zastosowanie dwóch dysków sieciowych, na których przechowywane są dwie kopie danych, umożliwia pracę systemu mimo awarii jednego z dysków sieciowych. Okres eksploatacyjny dla tej platformy szacowany jest na 3 lata. Na niezbędne prace konserwacyjne należy zabezpieczyć w budżecie przedsiębiorstwa 

\end{document}
