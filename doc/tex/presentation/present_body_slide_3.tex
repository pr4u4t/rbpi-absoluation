\begin{frame}
    \frametitle{Introduction}
    \begin{minipage}{\textwidth}
        Prolog is a general-purpose logic programming language associated with artificial intelligence and computational linguistics.
        Prolog has its roots in first-order logic, a formal logic, and unlike many other programming languages, Prolog is declarative: the program logic is expressed in terms of relations, represented as facts and rules. A computation is initiated by running a query over these relations.
        Prolog was one of the first logic programming languages, and remains the most popular among such languages today, with several free and commercial implementations available. The language has been used for theorem proving, expert systems, term rewriting, type inference, and automated planning, as well as its original intended field of use, natural language processing. Modern Prolog environments support the creation of graphical user interfaces, as well as administrative and networked applications.
        Prolog is well-suited for specific tasks that benefit from rule-based logical queries such as searching databases, voice control systems, and filling templates.
    \end{minipage}
\end{frame}
