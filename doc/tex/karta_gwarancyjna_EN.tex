\documentclass[10pt,a4paper]{article}

\usepackage{polski}
\usepackage[utf8]{inputenc}	
\usepackage[T1]{fontenc}
\usepackage{lmodern}
\usepackage{graphicx}
\usepackage{listings}
\usepackage{enumitem}
\usepackage{titlesec}
\usepackage{hyperref}
\usepackage{array}
\usepackage[margin=1in]{geometry}

%\usepackage{booktabs}

%variables
\newcommand{\shopname}{SHOP NAME}
\newcommand{\companyname}{COMPANY NAME}
\newcommand{\regon}{REGON}
\newcommand{\nip}{NIP}
\newcommand{\httpaddr}{SITE ADDRESS}
\newcommand{\address}{ADDRESS}
\newcommand{\mail}{MAIL ADDRESS}
\newcommand{\phone}{PHONE NUMBER}
\newcommand{\currency}{CURRENCY}

% use paragraph sign with numbering and break section title below
\titleformat{\section}[display]
	{\normalfont\Large\bfseries\filcenter}{\S\thesection}{1em}{}

% display subsection numbering without newline
\titleformat{\subsection}[runin]
	{\fontsize{12}{15}\filcenter\sffamily}
	{\thesubsection}{1em}{}

% command to set text left and right margin
\def\changehmargin#1#2{\list{}{\rightmargin#2\leftmargin#1}\item[]}
\let\endchangehmargin=\endlist 

% pdf form input adjustment
\newlength{\LabelWidth}%
\setlength{\LabelWidth}{1.3in}%
\newcommand*{\AdjustInputSize}[2][l]{\makebox[\LabelWidth][#1]{#2}}%
\def\textfieldwidth{2.25in}%

%define new column type
\newcolumntype{L}[1]{>{\raggedright\let\newline\\\arraybackslash\hspace{0pt}}m{#1}}
\newcolumntype{C}[1]{>{\centering\let\newline\\\arraybackslash\hspace{0pt}}m{#1}}
\newcolumntype{R}[1]{>{\raggedleft\let\newline\\\arraybackslash\hspace{0pt}}m{#1}}

%title autor etc

\title{OGÓLNE WARUNKI GWARANCJI} % The article title
\author{}
\date{} 

%----------------------------------------------------------------------------------------

\begin{document}
	\maketitle
 
	\section{SCOPE OF APPLICATION}
	
	\subsection {Overall Warranty Terms (OWT) constitute integral part of  Sales Agreements including rendered services agreements that are concluded by and between \shopname and the buyers purchasing products offered by this partnership if the agreements are not allowed to rule otherwise. Meaning of OWT terms used in further part:}
	
\begin{itemize}

\item \shopname \companyname \nip \regon
	
\item “Buyer”- a contractor purchasing products or services. This OWT is used only to contractors (entrepreneurs in accordance with an article 431of civil code) who are not consumers in accordance with an article 221 of c.c.) 

\item “Parties” –\ shop name and the buyer.

\item “OWT “– present Overall Warranty Terms \shopname. 

\item “Producer’s Warranty” – warranty conditions that are offered regardless of  present OWT, directly by the producer.

\item “Product” – products, goods and services which constitute the subject of statutory business activity\ shop name and in this scope are encompassed within warranty in Poland, in particular: furniture, chairs, floor-coverings etc. 

\item “Small volume products” – products that do not exceed the following sizes: 
Width 80 cm, depth 80 cm, height 140 cm and weight 50 kilos, and which do not overburden excessively a buyer during the transport e.g. a chair, a chest of drawers, a small cupboard or office equipment etc.

\item “Big volume products” – products that exceed the weight and size of small volume products or the one which, regardless their weight and size, were assembled into a set e.g. a desk, sets of cabinets and desks, a fitted wardrobe, kitchen furniture or a reception counter etc.

\item “Producer” – a manufacturer of products and goods which are sold via \shopname.

\item “Sales agreement” – a document under which \shopname realized delivery of products for the buyer.

\item “Complaint” – an in-writing application of product’s defect or damage.

\item“Place of assembly” – a buyer’s seat or another place, specified within the sales agreement.

\end{itemize}


\subsection{In accordance with this OWT the buyer is given with warranty on all the products sold via \ shop name. In addition, shop name\ can guarantee efficient functioning of offered products providing that they are in accordance with intended use and operation terms specified by producer’s documentation and present OWT.}

\subsection{In accordance with an article 558 § of civil code, a guarantee for the product is excluded.}

\section{Operation terms}

\subsection{Products should be used in dry and airy rooms of constant temperature and humidity}

\subsection{Products must be guarded against direct action of sunrays.}

\subsection{Leather products should be kept away from radiators or other sources of heat, with a minimal distance of 30 cm.}

\subsection{Leather maintenance must be performed with the use of special measures which form a protective coating resistant to water and oily substances.}

\subsection{Maintenance of furniture covered with artificial veneers should be performed with a soft cloth or a sponge, and with the use of washing and preservative preparations for wood-based, varnished and laminated surfaces (e.g. a non-ammonia window cleaning liquid).}

\subsection{Furniture covered with natural veneers should be cleaned so gently, with a soft cloth soaked with (never spill the liquid directly on furniture surface) 2- percent alcohol solution liquid (e.g. a non-ammonia window cleaning liquid, a soap solution in proportion of 250 ml soap flakes on 1 liter of warm water in combination with veneer maintenance milk except greasy preventive measures e.g. Pronto anti-dust); in accordance with tree rings direction.}

\subsection{Daylight can affect on natural veneer. During first 6-8 weeks when a piece of furniture is going to absorb the daylight one should not place any objects or decorations on the surface. Uneven daylight absorption can cause permanent discolorations, e.g. in shape of circles.}

\subsection{During natural veneer furniture operation it is recommended to use pads, especially laptop or keyboard pads, mouse mats, as well as on parts particularly exposed on mechanical damage}

\subsection{Furniture can be only curried over. (Note: it is not allowed to shift the furniture on the floor.)}

\subsection{Sets of cabinets and desks linked structurally must be dismantled before moving. It is recommended to be performed by the service team \shopname}

\subsection{Buyers perform cleaning and maintenance on their own and at their own expense.}

\section{WARRANTY COVERAGE}

\subsection{\shopname grants the buyer a guarantee on all the sold products and ensures their efficient functioning providing that they are in accordance with intended use and operation terms specified by producer’s documentation.}

\subsection {During the warranty period \shopname is obligated to free delivery of spare parts and also to service repairs.}
\subsection{ \shopname is responsible to the buyer only for physical defects caused by the reasons inherent in sold product. The warranty does not cover the defects arising from other causes, especially as a result of:}

\begin{itemize}

\item mechanical damages which can cause the defects during transport and transshipment (does not apply to carriage made by \shopname
\item damages resulting from the use of the product under conditions not corresponding to normal operating conditions, or arising from improper maintenance.
\item unauthorized modifications.
\item not proper assembly, maintenance, product storage.
\item damages caused by random events or other factors like fire, flood or lightning discharges etc.
\item unauthorized repairs

\end{itemize}

\subsection{The warranty does not cover parts that are subject to normal use resulting from its proper operation}
\subsection{The warranty does not cover changes due to sunlight nor the natural characteristics of the raw material such as: differences in Shades, glazing of veneered furniture, differences in textures and shades, scars and traces of insect stings in leather-covered products}
\subsection{The warranty does not cover the Product, which, on the basis of the submitted documents and features, can not be identified as a Product purchased from\shopname.}

\section{GUARANTEE PERIOD}   

\subsection{The Guarantee Period commences from the date of purchase and the product you have purchased is covered by a warranty of 24  months.}
\subsection{The parties can agree on different periods.} 
\subsection{\shopname will perform the duties warranty only upon presentation of the VAT invoice  or the contract of sale by the Buyer.
The Warranty Card will be issued  to the buyer upon his demand.} 
\subsection{If the Producer offers  longer warranty period, the Buyer has the right to use Overall Warranty Terms stated herein or the Manufacturer’s Warranty except for points 5.13 and 5.14.}

\section{WARRANTY AND NOTIFICATION PROCEDURE}

\subsection{These conditions must be fulfilled for the complaint to be valid:}

\begin{itemize}

\item The Buyer will deliver a written form notification ( a registered letter or an e-mail).
You can also submit your complaint by fax (tel. 61 843 02 40), using the Complaint Form -containing the name of product, its catalogue number, the date of purchase, detailed description of the damage and some additional information about the product’s defects. The photos of damages must also be included. The Complaint Form template can be found here: http://www.\shopname.com.pl/gwarancja/  or at the Company’s Office.
\item It is necessary to show the  purchase invoice  or the bill of sale of the faulty good.
\item You can deliver the item in person or  by courier directly to  the Company’s Office (small-sized products) or you can facilitate at the request of \shopname an access to the big-sized items  at the installation site.

\end{itemize}

\subsection{The defective product flaws revealed during the guarantee period should be reported to \shopname immediately, but no later than within 7 days from the date on which such defect is revealed.}

\subsection{The defective item should  be excluded from use or the Warranty is invalid.}

\subsection{\shopname will analyze the defectiveness of the item  within 14 working days from the moment the complaint was reported. If defectiveness lies in the product itself, the complaint shall be accepted and the replacement of the item or its repair procedure will take place .}

\subsection{If \shopname is not able to determine the defectiveness of the item itself, the product will be sent to the manufacturer. If the complaint is rejected by the manufacturer the item will be sent to the  installation site at buyer’s cost and risk.
 \shopname reserves the right to charge the Buyer with any additional fees connected with the out-of-warranty repair costs of the defective item.}

\subsection{Defects revealed during the guarantee period  will be repaired  free of charge no later than within 14 days after the written notification is provided. \shopname reserves the right to prolong the repair time /up to 21 days/  in case when there is a need to  replace the defective parts of an item. The Buyer will be informed about each step of the procedure.
The time needed to repair the defective item automatically prolongs the time of a specific product’s Warranty.}


\subsection{\shopname can replace the faulty good or return money if the defect is impossible to remove or due to escalating repair costs.
If the faulty product is replaced, warranty is renewed.}

\subsection{ Small-sized items can be sent back at buyer’s cost and risk after informing about this fact \shopname,
Accepting the complaint is understood as the replacement or repair of the faulty item. Handling costs will be returned to the Buyer.}

\subsection{Big-sized products can be repaired by \shopname at the installation site, depending on the kind of defect. If it is impossible, the product will be sent to the Manufacturer. The buyer will not be charge with handling costs.}

\subsection{The Buyer is obliged to facilitate the access to the faulty item at the installation site if the repair is needed. This also applies if the item must be disassembled or serviced.}

\subsection{Making a complaint does not mean that the Buyer  is exempted from the charges for the product. \shopname reserves the right to suspend the warranty claim if the Buyer falls behind with a payment /more than 7 days/.}
\subsection{The regulations stated hereby do not exclude the Buyer from the right to use the Warranty given by the  Manufacturer and offered directly by the factory producing the specific item.}

\subsection{In the case mentioned, all the regulations and warranties given by the Manufacturer start to apply. If the Buyer decides to make use of the Warranty given by the Manufacturer of the product,  the Warranty granted by \shopname is invalid.}

\subsection{In case of unsettled matters, the regulations of the civil code apply.}

\end{document}





