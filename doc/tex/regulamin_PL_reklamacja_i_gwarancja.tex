
			\subsection{} Konsument ma prawo odstąpić od umowy w terminie 14 dni bez podania jakiejkolwiek przyczyny.
			
			\subsection{} Termin do odstąpienia od umowy wygasa po upływie 14 dni od dnia w którym Konsument wszedł	w posiadanie rzeczy lub w którym osoba trzecia inna niż przewoźnik i wskazana przez Konsumenta weszła w posiadanie rzeczy
			
			\subsection{} Aby skorzystać z prawa odstąpienia od umowy, Konsument musi poinformować Przedsiębiorcę o swojej decyzji o odstąpieniu od umowy w drodze jednoznacznego oświadczenia - korzystając z danych Przedsiębiorcy podanych w niniejszym regulaminie.

			\subsection{} Konsument może skorzystać z wzoru formularza odstąpienia od umowy zamieszczonego na końcu niniejszego regulaminu, jednak nie jest to obowiązkowe.

			\subsection{} Aby zachować termin do odstąpienia od umowy, wystarczy, aby Konsument wysłał informację dotyczącą wykonania przysługującego prawa odstąpienia od umowy przed upływem terminu do odstąpienia od umowy.
	
			\subsection{} Skutki odstąpienia od umowy:
				\begin{itemize}
					\item w przypadku odstąpienia od umowy Przedsiębiorca zwraca Konsumentowi wszystkie otrzymane od Konsumenta płatności, w tym koszty dostarczenia rzeczy (z wyjątkiem: dodatkowych kosztów wynikających z wybranego przez Konsumenta sposobu dostarczenia innego niż najtańszy zwykły sposób dostarczenia oferowany przez Przedsiębiorcę w Sklepie), niezwłocznie, a w każdym przypadku nie później niż 14 dni od dnia, w którym Przedsiębiorca został poinformowany o decyzji Konsumenta o wykonaniu prawa odstąpienia od umowy.;

					\item  zwrotu płatności Przedsiębiorca dokona przy użyciu takich samych sposobów płatności, jakie zostały przez Konsumenta użyte w pierwotnej transakcji, chyba że Konsument wyraźnie zgodził się na inne rozwiązanie; w każdym przypadku Konsument nie poniesie żadnych opłat w związku z tym zwrotem;

					\item  Przedsiębiorca może wstrzymać się ze zwrotem płatności do czasu otrzymania towaru lub do czasu dostarczenia mu dowodu jego odesłania, w zależności od tego, które zdarzenie nastąpi wcześniej;

					\item  Konsument powinien odesłać towar na adres Przedsiębiorcy podany w niniejszym regulaminie niezwłocznie, a w każdym razie nie później niż 14 dni od dnia, w którym poinformował Przedsiębiorcę o odstąpieniu od umowy. Termin zostanie zachowany, jeżeli Konsument odeśle towar przed upływem terminu 14 dni;

					\item  Konsument ponosi bezpośrednie koszty zwrotu rzeczy;

					\item  Konsument odpowiada tylko za zmniejszenie wartości rzeczy wynikające z korzystania z niej w sposób inny niż było to konieczne do stwierdzenia charakteru, cech i funkcjonowania rzeczy.

				\end{itemize}
				
			\subsection{} W przypadku gdy ze względu na swój charakter rzeczy nie mogą zostać w zwykłym trybie odesłane pocztą informacja o tym, a także o kosztach zwrotu rzeczy, będzie się znajdować w opisie rzeczy w Sklepie.
 
