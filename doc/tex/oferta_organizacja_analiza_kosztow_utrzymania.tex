\par Na koszty utrzymania systemu składać się będą koszty prądu oraz koszty usług telekomunikacyjnych. Jako koszty telekomunikacyjne rozumiane są wydatki na łącze internetowe, oraz abonament za dwa numery telefoniczne polski i kanadyjski.
		
			\paragraph{Koszty prądu}
				\par Koszty prądu można wyliczyć posługując się danymi ze specyfikacji technicznej sprzętu komputerowego. Zużycie prądu w watach przedstawia poniższa lista:

				\begin{itemize}
					\item{Procesor serwerowy} 85W
			 
                    \item{Procesor komputerowy} 55W
			 
					\item{Płyta główna serwerowa} 70W
			
                    \item{Płyta główna komputerowa} 45W
			
					\item{Pamięć operacyjna} 4W
			
					\item{Dysk twardy HDD} 8W
			
					\item{Dysk twardy SSD} 3W
			
					\item{Karta graficzna} 15W
				\end{itemize}

				\par Co po przemnożeniu przez ilość elementów określonych w wycenie daje:
				
				\begin{itemize}
					\item{Procesor} 280W
			 
					\item{Płyta główna} 230W
			
					\item{Pamięć operacyjna} 24W
			
					\item{Dysk twardy HDD} 32W
			
					\item{Dysk twardy SSD} 9W
			
					\item{Karta graficzna} 45W
				\end{itemize}
				
				\par Daje to sumaryczne zużycie prądu przez sprzęt komputerowy 620W. Należy wziąć tu jeszcze pod uwagę sprawność zasilacza, która dla wybranego modelu wynosi ok. 80\%. Znaczy to, że sprzęt komputerowy poprzez straty na zasilaczu będzie pobierał 20\% energii więcej co daje około 744W. Wliczając w to pobór prądu przez urządzenia takie jak kasa fiskalna, router, drukarka, switch daje ok 800W, Biorąc pod uwagę pracę systemu 24 godziny na dobę przez cały rok daje zużycie energii na poziomie 7000 kWh rocznie. Do tych kosztów należy doliczyć jeszcze energię zużytą przez komputer na stanowisku pracowniczym pracujący 8 godzin dziennie przez 250 dni, (tyle bowiem średnio jest dni pracujących w roku) pobierający około 80W energii na godzinę. Daje to zużycie prądu rocznie na poziomie 160 kWh. Na tej podstawie można oszacować średnie roczne zużycie prądu 7150 kWh. Czyli w ujęciu miesięcznym 595 kWh, co biorąc pod uwagę średni koszt kWh prądu na Dolnym Śląsku około 0.55 zł/kWh daje szacunkowo 320zł/m-c. Zakładając 5\% niedoszacowanie wartości wychodzi około 340zł/m-c. Odchylenie to wynika z szacowania poboru prądu dla urządzeń w stanie pracy bez obciążenia. Nie będzie to już w tym wypadku wartość znacząca jednak warto o pamiętać o tym fakcie.

			\paragraph{Koszty telekomunikacyjne}
				\par Do kosztów telekomunikacyjnych należy wliczyć koszt utrzymania połączenia internetowego, ze względu na konieczność utrzymywania dwóch adresów IP co wynika wprost z rfc1035, określającego parametry łączą dla przechowywania domeny, daje 110zł/m-c. Dla łączą o parametrach 2 x 100/100Mbps. Do kosztów tych należy doliczyć jeszcze koszt dwóch numerów w Polsce i Kanadzie z nielimitowanym planem taryfowym 70zł/m-c przy wykorzystaniu technologii VOIP (Voice Over IP).
		
			\paragraph{Porównanie kosztów}
				\par Koszty prądu na same tylko 3 serwery wynoszą około 280 zł/m-c doliczając do tego koszt łącza internetowego w wysokości 110zł/m-c otrzymujemy 390zł miesięcznie  dla porównania w tej samej cenie możemy np. w serwerowniach takich jak ovh.pl czy home.pl wynająć jeden 4 rdzeniowy komputer z 1 terabajtowym dyskiem twardym. Wynajęcie równoważnej infrastruktury to koszt około 1500zł/m-c. Co przemawia stanowczo za utworzeniem własnego zaplecza technicznego. 
 
